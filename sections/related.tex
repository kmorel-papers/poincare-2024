\vspace{-.1in}
\section{Related Work}
\label{sec:related}

\subsection{\poincare Maps}
\poincare maps are an important tool developed for the study of dynamical systems. One of the key advantages is that it provides a dimension reduction to study periodic systems~\cite{morrison2000magnetic}. In the case of our fusion example, this reduction is from 3D to 2D.  L{\"o}effelmann et al.~\cite{Loeffelmann-1997-VPM} provide a summary of the usage of \poincare maps for visualization across a variety of application areas and demonstrate their usefulness in flow analysis.
\poincare maps have been applied to fusion in a number of works, including the following~\cite{Sanderson2010, Sanderson2012understanding, Tricoche2011 }.
%Techniques for computing \poincare maps will be discussed in the following section.


\subsection{Streamlines and Particle Advection}

A widely used visualization algorithm for flow fields is a technique called streamlines \cite{VTKTextbook}.
The technique starts with a velocity vector field representing the flow of a fluid at each point in the domain, which is a common output from computational fluid dynamics simulations.
This fluid is visually represented by one or more curves that trace the trajectory of that part of the flow.
This is modeled and computed as a massless particle instantaneously moving with the velocity determined by the field at the particle's position.
The particle is placed at a seed point and then is advected by pushing it by the vector field.
The computation (described in more detail in Section~\ref{sec:method}) solves a differential equation to find the curve that is tangent to the vector field everywhere.

Although these traces of advected particles can be visualized directly as streamlines, they also form the basis of numerous visualization algorithms, some of which require the advection of a great many particles \cite{Hultquist1992,Guo2016}.
Because particle advection is the greatest computational cost of these algorithms, much research has been invested in optimizing this process \cite{Zhang2018, Yenpure2023}.
The work in this paper leverages the particle advection provided by the VTK-m library \cite{moreland2016vtk}.
VTK-m provides a flexible particle advection algorithm that is optimized for a variety of processors \cite{Pugmire2018}.

Technically, a magnetic field is not a flow field; it does not describe the movement of matter.
However, we are interested in extracting magnetic field lines: curves that are tangent to the magnetic field everywhere.
This is the same property a particle advection trajectory has to its velocity field, and thus it is valid to extract these magnetic field lines by treating magnetism as a flow and leveraging the aforementioned particle advection algorithms.
This paper often refers to flowing particles even though that does not match the physical process.