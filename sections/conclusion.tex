\section{Summary and Future Work}
\label{sec:conclusion}
In this paper, we have described a collaborative effort among computer scientists and physicists to develop a new tool for performing analysis and visualization of the magnetic fields in fusion devices. \poincare\ plots are notoriously expensive to generate due to high computational costs. Because of the complexity of the magnetic fields in tokamaks, the definition of the vector field needed for generating field lines is often code-specific and complex. Working together with the XGC physics team, we developed a tool that reproduces the complex definition of the magnetic field and uses the concepts in \vtkm to efficiently map the computation onto both multi-core CPU and GPU devices.  The development of this new tool has made it possible for physicists to perform unprecedented analysis on XGC simulations running on some of the most powerful supercomputers in the world. Previously, it was only possible to perform \poincare\ analysis on a limited number of timesteps from a simulation. This work has made it possible to generate \poincare\ plots in situ for every time step of the simulation. This provides the XGC team with a powerful new tool to study the physics of burning plasmas.

Although we have achieved significant speedups in computing \poincare maps, there is still room for additional performance. The bottleneck in the algorithm is the cost of cell location. There are two major ways to address this problem. The first is to avoid performing cell locations using an adaptive step-sized solver. These techniques improve over the brute-force nature of fixed step-size methods to adjust according to the local changes in the vector field. We plan to explore the use of these solvers to increase performance while maintaining accuracy.
The second is to improve the performance of the cell location on GPUs. The costs for memory access can vary dramatically on the GPU depending on where the memory is located. We plan to explore methods to achieve better caching of data in faster memory to improve overall performance.

%\ken{One thing that I think is missing from this paper is at least one image of a \poincare plot that shows its utility. Figure \ref{fig:result} has results, but they are boring. Can we change this to conclusions and show a picture with a \poincare plot with interesting features? Maybe homoclinic tangles. Maybe just islands.}

